\documentclass[11pt]{beamer}

%% Beamer Options
\setbeamertemplate{footline}[frame number]
\setbeamertemplate{navigation symbols}{}
\setbeamertemplate{caption}[numbered]

\usefonttheme{serif}

%% Packages
\usepackage[french]{babel}
\usepackage{graphicx}
\usepackage{tikz}
\usepackage{subcaption}
\usepackage[T1]{fontenc}
\usepackage[autolanguage]{numprint}
\usepackage{hyphenat}
\usepackage{amsmath}
\usepackage{siunitx}
\usepackage{unicode-math}
\usepackage[european, american inductors, straightvoltages]{circuitikz}
\usepackage{lmodern}
\hyphenation{}

\usepackage{caption}
\captionsetup[figure]{name=Fig.}

\sisetup{%
  mode = math,
  detect-family,
  detect-weight,  
  exponent-product = \cdot,
  number-unit-separator=\text{\,},
  output-decimal-marker={\text{,}},
}



\tikzset{
  invisible/.style={opacity=0},
  visible on/.style={alt={#1{}{invisible}}},
  alt/.code args={<#1>#2#3}{%
      \alt<#1>{\pgfkeysalso{#2}}{\pgfkeysalso{#3}} % \pgfkeysalso doesn't change the path
    },
}


\makeatletter
\def\ifemptyarg#1{%
  \if\relax\detokenize{#1}\relax % H. Oberdiek
    \expandafter\@firstoftwo
  \else
    \expandafter\@secondoftwo
  \fi}
\makeatother

\let\oldcaption\caption
\AtBeginDocument{%
  \renewcommand{\caption}[2][]{%
    \ifemptyarg{#2}{}{\oldcaption[#1]{#2}}%
  }%
}

%% Mise en page
% \newcommand{\border}[1]{\bbox[10px, border: 1px solid]{#1}}
\newcommand{\figcentered}[2]{
  \begin{figure}[ht]
    \centering
    \includegraphics[width=\textwidth,height=0.7\textheight,keepaspectratio]{#1}
    \caption{#2}
  \end{figure}
}
\newcommand{\figbesides}[4]{
  \begin{figure}[!tbp]
    \centering
    \begin{minipage}[b]{0.49\textwidth}
      \begin{figure}
        \includegraphics[width=\textwidth]{#1}
        \caption{#2}
      \end{figure}
    \end{minipage}
    \hfill
    \pause
    \begin{minipage}[b]{0.49\textwidth}
      \begin{figure}
        \includegraphics[width=\textwidth]{#3}
        \caption{#4}
      \end{figure}
    \end{minipage}
  \end{figure}
}
\newcommand{\figbesidestextLR}[2]{
  \begin{tabular}{cl}
    \begin{tabular}{c}
      \includegraphics[height=\textheight,width=0.5\linewidth,keepaspectratio]{#1}
    \end{tabular}
     & \begin{tabular}{l}
         \parbox{0.4\linewidth}{#2}
       \end{tabular}
  \end{tabular}
}
\newcommand{\figbesidestextRL}[2]{
  \begin{tabular}{lc}
    \begin{tabular}{l}
      \parbox{0.4\linewidth}{#1}
    \end{tabular}
     & \begin{tabular}{c}
         \includegraphics[height=\textheight,width=0.5\linewidth,keepaspectratio]{#2}
       \end{tabular}
  \end{tabular}
}
\newcommand{\figquad}[8]{
  \begin{figure}[ht]
    \onslide<1->
    \begin{minipage}[b]{0.5\linewidth}
      \centering
      \begin{figure}
        \includegraphics[width=.8\linewidth]{#1}
        \caption{#2}
      \end{figure}
      \vspace{1ex}
    \end{minipage}%%
    \onslide<2->
    \begin{minipage}[b]{0.5\linewidth}
      \centering
      \begin{figure}
        \includegraphics[width=.8\linewidth]{#3}
        \caption{#4}
      \end{figure}
      \vspace{1ex}
    \end{minipage}
    \onslide<3->
    \begin{minipage}[b]{0.5\linewidth}
      \centering
      \begin{figure}
        \includegraphics[width=.8\linewidth]{#5}
        \caption{#6}
      \end{figure}
      \vspace{1ex}
    \end{minipage}%% 
    \onslide<4->
    \begin{minipage}[b]{0.5\linewidth}
      \centering
      \begin{figure}
        \includegraphics[width=.8\linewidth]{#7}
        \caption{#8}
      \end{figure}
      \vspace{1ex}
    \end{minipage}
  \end{figure}
}
\newcommand{\figdiagTRBL}[6]{
  \begin{figure}[ht]
    \onslide<1->
    \begin{minipage}[b]{0.5\linewidth}
      \centering
      \begin{figure}
        \includegraphics[width=.8\linewidth]{#1}
        \caption{#2}
      \end{figure}
      \vspace{1ex}
    \end{minipage}%%
    \onslide<2->
    \raisebox{15ex}{
      \begin{minipage}{0.5\linewidth}
        \centering
        {#3}
        \vspace{1ex}
      \end{minipage}
    }
    \onslide<4->
    \raisebox{15ex}{
      \begin{minipage}{0.5\linewidth}
        \begin{center}
          {#4}
        \end{center}
        \vspace{1ex}
      \end{minipage}}%% 
    \onslide<3->
    \begin{minipage}[b]{0.5\linewidth}
      \centering
      \begin{figure}
        \includegraphics[width=.8\linewidth]{#5}
        \caption{#6}
      \end{figure}
      \vspace{1ex}
    \end{minipage}
  \end{figure}
}
\newcommand{\figdiagTLBR}[6]{
  \begin{figure}[ht]
    \onslide<2->
    \raisebox{15ex}{
      \begin{minipage}{0.5\linewidth}
        \centering
        {#1}
        \vspace{1ex}
      \end{minipage}
    }%%
    \onslide<1->
    \begin{minipage}[b]{0.5\linewidth}
      \centering
      \begin{figure}
        \includegraphics[width=.8\linewidth]{#2}
        \caption{#3}
      \end{figure}
      \vspace{1ex}
    \end{minipage}
    \onslide<3->
    \begin{minipage}[b]{0.5\linewidth}
      \centering
      \begin{figure}
        \includegraphics[width=.8\linewidth]{#4}
        \caption{#5}
      \end{figure}
      \vspace{1ex}
    \end{minipage}%%
    \onslide<4->
    \raisebox{15ex}{
      \begin{minipage}{0.5\linewidth}
        \begin{center}
          {#6}
        \end{center}
        \vspace{1ex}
      \end{minipage}}
  \end{figure}
}



%% Caractères
\newcommand{\partialf}[2]{\frac{\partial{#1}}{\partial{#2}}}
\newcommand{\ceil}[1]{\left\lceil{#1}\right\rceil}
\newcommand{\floor}[1]{\left\lfloor{#1}\right\rfloor}
\newcommand{\paren}[1]{\left(#1\right)}
\newcommand{\bracket}[1]{\left[#1\right]}
\newcommand{\abs}[1]{\left\vert{#1}\right\vert}
